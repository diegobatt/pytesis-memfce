
\begin{abstract}
    Esta tesis se centra en el análisis de datos topológicos (TDA) y, específicamente, en el desarrollo de tests de hipótesis para detectar características topológicas de una variedad a partir de muestras i.i.d. de puntos. Se introducen conceptos clave de TDA y se describen técnicas actuales para realizar tests sobre diagramas de persistencia, los cuales resumen la homología de un conjunto de datos, facilitando la identificación de ciclos de primer grado (curvas no deformables a un punto). Tradicionalmente, estas pruebas se basan en la norma euclídea, pero aquí se explora el uso de la distancia de Fermat, que ha demostrado ser beneficiosa en conjuntos con topología subyacente. Mediante simulaciones de regiones de confianza sobre datos sintéticos y reales, se comparan ambas distancias, demostrando que la distancia de Fermat supera a la euclídea en la detección de características topológicas, con resultados estadísticamente significativos. Además, se muestra que la distancia de Fermat es computacionalmente escalable frente a la dimensionalidad de los datos. Como aporte adicional, se desarrolló una biblioteca en R para facilitar investigaciones futuras sobre el uso de la distancia de Fermat en este contexto.


    \textbf{Palabras clave:} TDA, topología, homología persistente, distancia de Fermat, prueba de hipótesis, diagramas de persistencia, regiones de confianza.

    \begin{flushleft}
        \vfill
        FIRMA DEL DIRECTOR \hfill FIRMA DEL MAESTRANDO
    \end{flushleft}

\end{abstract}

\newpage

\renewcommand{\abstractname}{Abstract}
\begin{abstract}

    This dissertation focuses on topological data analysis (TDA) and, specifically, on the development of hypothesis tests to detect topological features of a manifold based on i.i.d. samples. Key TDA concepts are introduced, and state-of-the-art techniques for hypothesis testing on persistence diagrams are described. These diagrams summarize the homology of a dataset, helping to identify first-degree cycles (curves that cannot be deformed to a point). Traditionally, such tests rely on the Euclidean norm, but this work explores the use of the Fermat distance, which has shown benefits in datasets with underlying topology. Through simulations of confidence sets on synthetic and real datasets, both distances are compared, demonstrating that Fermat distance outperforms Euclidean distance in detecting topological features, with statistically significant results. Moreover, Fermat distance is shown to be computationally scalable with respect to data dimensionality. As an additional contribution, an R library was developed to facilitate future research on Fermat distance in this context.


    \textbf{Keywords:} TDA, topology, persistent homology, Fermat distance, hypothesis testing, persistence diagramas, confidence sets.

    \begin{flushleft}
    \vfill

    DIRECTOR'S SIGNATURE \hfill STUDENT'S SIGNATURE
    \end{flushleft}

\end{abstract}
